\documentclass{article}
\usepackage{listings}
\usepackage{color}
\usepackage{fullpage} % Paquete para ajustar los márgenes
\usepackage{url}

\title{Métricas de Usabilidad de Software}
\author{Nelyda Ayde Condori Condori}
\date{June 13, 2024}

\definecolor{codegray}{gray}{0.9}

\lstset{
    backgroundcolor=\color{codegray},
    basicstyle=\ttfamily\footnotesize,
    breaklines=true,
    frame=single,
    tabsize=2,
    captionpos=b,
    keepspaces=true
}

\begin{document}

\maketitle

\section{Definición y Tipos}

Las métricas de usabilidad son esenciales para evaluar la efectividad y eficiencia
de un software. Algunos tipos importantes son:

\begin{itemize}
    \item \textbf{Eficiencia de Tarea:} Tiempo necesario para completar una tarea específica.
    \item \textbf{Tasa de Éxito:} Porcentaje de tareas completadas con éxito.
    \item \textbf{Errores:} Número de errores cometidos al realizar una tarea.
    \item \textbf{Satisfacción del Usuario:} Percepción del usuario sobre la facilidad de uso del software.
    \item \textbf{Aprendibilidad:} Facilidad con la que los nuevos usuarios pueden completar tareas básicas.
    \item \textbf{Memorabilidad:} Capacidad de los usuarios para recordar cómo usar el sistema después de un periodo de no usarlo.
\end{itemize}

\section{Aplicaciones y Limitaciones}

\subsection{Aplicaciones}

Las métricas de usabilidad pueden ser utilizadas para:

\begin{itemize}
    \item Evaluar y mejorar la UI.
    \item Comparar diferentes versiones del software.
    \item Evaluar prototipos antes de la implementación completa.
    \item Monitorear mejoras en la usabilidad.
\end{itemize}

\subsection{Limitaciones}

Algunas limitaciones incluyen:

\begin{itemize}
    \item Pueden ser subjetivas.
    \item Requieren la participación de usuarios finales.
    \item Pueden variar según el contexto de uso.
\end{itemize}

\section{Ejemplo de Implementación en Python}

Aquí se muestra un ejemplo de cómo implementar la medición de algunas métricas de usabilidad en Python:

\begin{lstlisting}[language=Python]
import time
import random

def realizar_tarea():
    tiempo_inicio = time.time()
    tarea_completada = random.choice([True, False])
    errores = random.randint(0, 3)
    time.sleep(random.uniform(1, 5))
    tiempo_fin = time.time()
    tiempo_total = tiempo_fin - tiempo_inicio
    return tiempo_total, tarea_completada, errores

usuarios = 10
tiempos = []
exitos = 0
errores_totales = 0

for _ in range(usuarios):
    tiempo, exito, errores = realizar_tarea()
    tiempos.append(tiempo)
    if exito:
        exitos += 1
    errores_totales += errores

eficiencia_promedio = sum(tiempos) / usuarios
tasa_exito = (exitos / usuarios) * 100
errores_promedio = errores_totales / usuarios

print(f"Eficiencia de Tarea Promedio: {eficiencia_promedio:.2f} segundos")
print(f"Tasa de éxito: {tasa_exito:.2f}%")
print(f"Errores Promedio por Usuario: {errores_promedio:.2f}")
\end{lstlisting}

\section{Encuestas de Satisfacción del Usuario}

Para evaluar la satisfacción del usuario, se puede usar una herramienta común
como el System Usability Scale (SUS). A continuación, se muestra un ejemplo de código Python para una encuesta de satisfacción:

\begin{lstlisting}[language=Python]
def encuesta_satisfaccion():
    preguntas = [
        "1. Me gustaría usar este sistema frecuentemente.",
        "2. Encontré el sistema innecesariamente complejo.",
        "3. Pensé que el sistema era fácil de usar.",
        "4. Creo que necesitaría la ayuda de una persona técnica para usar este sistema.",
        "5. Encontré que las varias funciones en este sistema estaban bien integradas.",
        "6. Pensé que había demasiada inconsistencia en este sistema.",
        "7. Imagino que la mayoría de las personas aprenderán a usar este sistema muy rápidamente.",
        "8. Encontré el sistema muy difícil de usar.",
        "9. Me sentí muy confiado usando el sistema.",
        "10. Necesité aprender muchas cosas antes de poder trabajar con este sistema."
    ]
    respuestas = []
    for pregunta in preguntas:
        respuesta = int(input(f"{pregunta} (1 = Totalmente en desacuerdo, 5 = Totalmente de acuerdo): "))
        respuestas.append(respuesta)

    return respuestas

usuarios = 5
resultados = []

for _ in range(usuarios):
    resultado = encuesta_satisfaccion()
    resultados.append(resultado)

promedios = [sum(col) / len(col) for col in zip(*resultados)]

print("Promedio de Satisfacción por Pregunta:")
for i, promedio in enumerate(promedios):
    print(f"Pregunta {i + 1}: {promedio:.2f}")
\end{lstlisting}

\section{Referencias}

\begin{enumerate}
    \item Johnson, D., Smith, A. (2019). Software Productivity Metrics in Agile Development. Journal of Software Engineering, 34(2), 123-134. doi: \url{10.1016/j.jsofteng.2019.01.004}.
    \item Fenton, N. E., Pfleeger, S. L. (1994). Software Metrics: A Rigorous and Practical Approach. PWS Publishing Co.
    \item Nielsen, J. (1993). Usability Engineering. Academic Press.
    \item ISO/IEC 25010:2011. Systems and software engineering – Systems and software Quality Requirements and Evaluation (SQuaRE) – System and software quality models. International Organization for Standardization.
    \item Brooke, J. (1996). SUS: A ’quick and dirty’ usability scale. In P. W. Jordan, B. Thomas, B. A. Weerdmeester, I. L. McClelland (Eds.), Usability Evaluation in Industry.
\end{enumerate}

\end{document}
